\documentclass[conference]{IEEEtran}

\usepackage[british]{babel}
\usepackage{graphicx}
\usepackage{paralist}
\usepackage[hyphens]{url}
\usepackage{enumerate}
\usepackage[noadjust]{cite}
\usepackage[pdftex,colorlinks=true]{hyperref}


% correct bad hyphenation here
%\hyphenation{op-tical net-works semi-conduc-tor}


\begin{document}
%
% paper title
\title{Rethinking Pedagogies for Programming:\\Apprentices, Codemanship and Software Carpentry}


% author names and affiliations
% use a multiple column layout for up to three different
% affiliations
% author names and affiliations
\author{
    \IEEEauthorblockN{Tom Crick\IEEEauthorrefmark{1},
      James H. Davenport\IEEEauthorrefmark{2} and Alan
      Hayes\IEEEauthorrefmark{2}}
    \IEEEauthorblockA{\IEEEauthorrefmark{1}Department of Computing,
      Cardiff Metropolitan University, UK
    \\tcrick@cardiffmet.ac.uk}
    \IEEEauthorblockA{\IEEEauthorrefmark{2}Department of Computer Science,
      University of Bath, UK
    \\\{j.h.davenport,a.hayes\}@bath.ac.uk}
}

% conference papers do not typically use \thanks and this command
% is locked out in conference mode. If really needed, such as for
% the acknowledgment of grants, issue a \IEEEoverridecommandlockouts
% after \documentclass


% use for special paper notices
%\IEEEspecialpapernotice{(Invited Paper)}


% make the title area
\maketitle

\begin{abstract}
The impact of the application of computational techniques across
science and engineering has fundamentally affected practices within
those disciplines. Computing is both a rigorous academic discipline in
its own right and also facilitates and supports a wide range of other
disciplines, from computational physics to computational social
science; in essence it has become a bridge for interdisciplinarity: it
now not only supports how science and engineering is done, but what
science and engineering is done.

Over the past five years of teaching programming, we have moved away
from focusing primarily on syntax, to developing a deeper
understanding of principles of programming, transferable language
semantics, underlying constructs and structures, as well as fostering
a culture of creating useful and usable software artefacts: in
summary, software carpentry and codemanship. This has a dual focus:
firstly, it develops a high-level appreciation for why we are teaching
programming -- essentially to solve real-world problems, using the
most appropriate languages and environments; secondly, it enables us
to embed the use of tools, methodologies and techniques so as to start
to develop best practice for real-world software
development. Programming should be viewed a craft, and is best learned
as a practical skill: seeing the master doing it, and doing it on
one's own, corrected by those more proficient. How best to foster this
culture of creating useful and usable software artefacts? Furthermore,
how does one ensure that the students actually engage in the process
and do the frequent practice required for this system to work?

In this paper, we explore and evaluate some of the pedagogies for
teaching programming — as well as practical solutions to these
questions — as developed at Cardiff Metropolitan University (UK) in
teaching introductory programming to computer science undergraduate
students, and as applied elsewhere, including to both mathematicians
and computer scientists at the University of Bath (UK).
\end{abstract}


\section{Introduction}\label{intro}

Cite~\cite{brown-et-al-toce2014}.
% bib
\bibliographystyle{IEEEtran}
\bibliography{educon2016}

\end{document}
